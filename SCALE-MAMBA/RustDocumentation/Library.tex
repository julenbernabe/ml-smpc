\mainsection{Standard Library}


\subsection{Boxes}
A box is essentially an array of size one. It is a `safe' way for users
to write/read from the SCALE memory one element at a time.
To load the standard library version of a Box use
\begin{lstlisting}
     use scale_std::heap::Box;
\end{lstlisting}
As it is tied to the SCALE memory it only comes in the
types  $\ints$, $\SecretI$, $\ClearModp$ and $\SecretModp$.

\func{Box::uninitialized()}
To initialise a box with no pre-defined value use
\begin{lstlisting}
    let mut a= Box<SecretI64>::uninitialized();
\end{lstlisting}

\func{a.set(x)}
Does the obvious.

\func{a.get()}
Again does the obvious.


\subsection{Arrays}
This data type is implemented using the SCALE memory.
It is currently available for the following types;
$\ints$, $\SecretI$, $\ClearModp$, $\SecretModp$,
\verb|ClearFixed<K,F>|, \verb|SecretFixed<K,F>|,
\verb|ClearFloat<V,P>|, \verb|SecretFloat<V,P>|,
\verb|ClearIEEE| and \verb|SecretIEEE|.
You can always use simple arrays of the form
\begin{lstlisting}
    let key128: [SecretI64; 2] = [zero, none];
\end{lstlisting}
as in the earlier AES example. But these will not compile as
soon as the size becomes too large. The simple arrays will
compile to registers, and thus be more efficient than arrays
stored in memory. But arrays stores in memory are more flexible.
To load the standard library version of Arrays use
\begin{lstlisting}
     use scale_std::array::Array;
\end{lstlisting}

\func{Array::uninitialized()}
To initialise an array with no pre-defined values use
\begin{lstlisting}
    let mut a= Array<SecretI64, 25>::uninitialized();
\end{lstlisting}


\func{a.fill(x)}
Fills an array, used with initialization
\begin{lstlisting}
    let a: Array<SecretI64, 25> = Array::fill(zero);
\end{lstlisting}


\func{a.set(i,x)}
Does the obvious, but has no bound checks, i.e. you can access the 50th element of a 25 element array.

\func{a.get(i)}
Get the reference of an element from the array by checking we do not make any memory overflow

\func{a.get_unchecked(i)}
Get the reference of an element from the array without checking we do not make any memory overflow

\func{a.get_mut(i)}
Get the mutable reference of an element from the array by checking we do not make any memory overflow.

\func{a.get_mut_unchecked(i)}
Get the mutable reference of an element from the array without checking we do not make any memory overflow

\func{drop(a)}
Deletes the array from memory.


\func{a.iter()}
Iterator over an \verb|Array|,
\begin{lstlisting}
    for (i, val) in a.iter().enumerate() {
        ....
    }
\end{lstlisting}
To iterate backwards use
\begin{lstlisting}
    for (i, val) in a.iter().rev().enumerate() {
        ....
    }
\end{lstlisting}
For more details on using iterators see \url{https://doc.rust-lang.org/std/iter/trait.Iterator.html}.

\func{a.addr(i)}
This returns the address in memory of the $i$-th element in the array.

\func{a.reverse()}
Returns the same array, but with the elements in the reverse order, i.e.
$a[i] = a[\ell-1-i]$ for $i=0,\ldots,\ell-1$.

\func{a.reveal()}
For the secret types this creates a non-secret version, as in.
\begin{lstlisting}
     let cint_array = sint_array.reveal();
\end{lstlisting}

\func{a.slice(range)}
Converts (part of) an array to a slice, of which the length is only known at runtime.
\begin{lstlisting}
    let a_slice: Slice<SecretI64> = a.slice(3..);
\end{lstlisting}

\subsubsection{Stack Operations on Arrays}
One can also apply the stack operations on an array, which
applies to all the elements in the array.
Note the pop operation works in reverse, so that pushing and
then popping gives the same array.
The stack read operations (pop, peek etc) will pop and peek
the entire array, so you need to ensure the stack really has
that many elements on it; i.e. no access is performed outside
the stack, otherwise a runtime crash will occur.

\func{push(x), pop()}
\func{peek(i), poke(i, x)}
\func{peek_from_top(i), poke_from_top(i, x)}
These are called using the following convention
\begin{lstlisting}
    unsafe { Array::push(&a) }
    let a = unsafe { Array::<SecretI64, 25>::pop() };
\end{lstlisting}

\subsection{Slices}
\label{sec:slices}
This data type is equal to Arrays, the only difference being that the length of a slice is not known at compile time.
Just like Arrays, this data type is implemented using the SCALE memory.
It is currently available for the following types
$\ints$, $\SecretI$, $\ClearModp$, $\SecretModp$,
\verb|ClearFixed<K,F>|, \verb|SecretFixed<K,F>|,
\verb|ClearFloat<V,P>|, \verb|SecretFloat<V,P>|,
\verb|ClearIEEE| and \verb|SecretIEEE|
data types.
To load the standard library version of slices use
\begin{lstlisting}
     use scale_std::slice::Slice;
\end{lstlisting}

\func{Slice::uninitialized()}
To initialise a slice with no pre-defined values use
\begin{lstlisting}
    let mut a= Slice<SecretI64>::uninitialized();
\end{lstlisting}

\func{s.set(i,x)}
Does the obvious.


\func{a.get(i)}
Get the reference of an element from the slice by checking we do not make any memory overflow

\func{a.get_unchecked(i)}
Get the reference of an element from the slice without checking we do not make any memory overflow

\func{a.get_mut(i)}
Get the mutable reference of an element from the slice by checking we do not make any memory overflow.

\func{a.get_mut_unchecked(i)}
Get the mutable reference of an element from the slice without checking we do not make any memory overflow

\func{drop(a)}
Deletes the slice from memory.


\func{s.iter()}
Iterator over an \verb|Slice|,
\begin{lstlisting}
    for (i, val) in s.iter().enumerate() {
        ....
    }
\end{lstlisting}
To iterate backwards use
\begin{lstlisting}
    for (i, val) in a.iter().rev().enumerate() {
        ....
    }
\end{lstlisting}

\func{s.addr(i)}
This returns the address in memory of the $i$-th element in the slice.

\func{s.reverse()}
Returns the same slice, but with the elements in the reverse order, i.e.
$s[i] = s[\ell-1-i]$ for $i=0,\ldots,\ell-1$.

\func{s.reveal()}
For the secret types this creates a non-secret version, as in.
\begin{lstlisting}
     let cint_slice = sint_slice.reveal();
\end{lstlisting}

\func{s.slice(range)}
Returns part of the original slice.
\begin{lstlisting}
    let s_slice: Slice<SecretI64> = s.slice(3..);
\end{lstlisting}

\func{s.len()}
Returns the length of the slice as a u64-number.
\begin{lstlisting}
    let length: u64 = s.len();
\end{lstlisting}

\subsection{Arithmetic on Arrays and Slices}
These operations only apply to special types of Arrays or Slices.

\subsubsection{Private IO}
Since private IO requires rounds of communication we provide helper functions
for Arrays and Slices which use a constant number of rounds, irrespective of the
size of the Array or Slice.

\func{Array<SecretModp,N>::private_input(p,c)}
\func{a.private_output(p,c)}
\func{Slice<SecretModp>::private_input(N, p,c)}
\func{s.private_output(p,c)}
These are all called in the following manner
\begin{lstlisting}
    let a: Array<SecretModp, 5>=Array::private_input(1, 10);
    a.private_output(2, 10);


    let b: Slice<SecretModp>=Slice::private_input(6, 1, 10);
    b.private_output(2, 10);
\end{lstlisting}

\subsubsection{Addition/Subtraction/Multiplication}
Vectorized addition, subtraction, multiplication, division and modular-remainder
of Arrays and Slices of type $\ClearModp$ and $\SecretModp$ 
are implemented via either an operator notation, or a member function notation.
These are not implemented for secret-secret multiplication and the 
division/modular-remainder are only implemented for clear-clear operations.
The operator variants are often slower, as they can require a \verb|clone|
before calling (depending on the usage of the arguments in other parts of
the code).
The member function variants never require a clone.
We present an example for Arrays, an equivalent syntax applies to Slices,
\begin{lstlisting}
    const N: u64=200;
    let mut ca_arr: Array<ClearModp, N> = Array::uninitialized();
    let mut cb_arr: Array<ClearModp, N> = Array::uninitialized();
    let mut sa_arr: Array<SecretModp, N> = Array::uninitialized();
    let mut sb_arr: Array<SecretModp, N> = Array::uninitialized();

    ...

    // Operator notation (V slow)
    let a_add_cc = ca_arr.clone() + cb_arr.clone();
    let a_add_cs = ca_arr.clone() + sb_arr.clone();
    let a_add_sc = sa_arr.clone() + cb_arr.clone();
    let a_add_ss = sa_arr.clone() + sb_arr.clone();
    let a_sub_cc = ca_arr.clone() - cb_arr.clone();
    let a_sub_cs = ca_arr.clone() - sb_arr.clone();
    let a_sub_sc = sa_arr.clone() - cb_arr.clone();
    let a_sub_ss = sa_arr.clone() - sb_arr.clone();
    let a_mul_cc = ca_arr.clone() * cb_arr.clone();
    let a_mul_cs = ca_arr.clone() * sb_arr.clone();
    let a_mul_sc = sa_arr.clone() * cb_arr.clone();
    let a_div_cc = ca_arr.clone() / cb_arr.clone();
    let a_mod_cc = ca_arr.clone() % cb_arr.clone();

    
    // Operator notation (slow)
    let a_add_cc = &ca_arr + cb_arr.clone();
    ...
    let a_mod_cc = &ca_arr % cb_arr.clone();


    // Operator notation (slow)
    let a_add_cc = ca_arr.clone() + &cb_arr;
    ...
    let a_mod_cc = ca_arr.clone() % &cb_arr;


    // Operator notation (fastest)
    let a_add_cc = &ca_arr + &cb_arr;
    ...
    let a_mod_cc = &ca_arr % &cb_arr;


    // Member function notation (fastest)
    let a_add_cc_t = ca_arr.add_clear(&cb_arr);
    let a_add_cs_t = ca_arr.add_secret(&sb_arr);
    let a_add_sc_t = sa_arr.add_clear(&cb_arr);
    let a_add_ss_t = sa_arr.add_secret(&sb_arr);
    let a_sub_cc_t = ca_arr.sub_clear(&cb_arr);
    let a_sub_cs_t = ca_arr.sub_secret(&sb_arr);
    let a_sub_sc_t = sa_arr.sub_clear(&cb_arr);
    let a_sub_ss_t = sa_arr.sub_secret(&sb_arr);
    let a_mul_cc_t = ca_arr.mul_clear(&cb_arr);
    let a_mul_cs_t = ca_arr.mul_secret(&sb_arr);
    let a_mul_sc_t = sa_arr.mul_clear(&cb_arr);
    let a_div_cc_t = ca_arr.div_clear(&cb_arr);
    let a_mod_cc_t = ca_arr.mod_clear(&cb_arr);
\end{lstlisting}
The slice variants do not check whether the two source arrays have
the same number of elements, the size is always taken to be that
of \verb|*this|.

\subsubsection{Advanced Arithmetic on Arrays and Slices}
\func{a.evaluate(c)}
Given an array or slice $a$ of length $\ell$ of type $\ClearModp$ or
$\SecretModp$ this evaluates the polynomial $\sum_{i=0}^{\ell-1} a_i \cdot c^i$
for the $\ClearModp$ value $c$.

\func{Array<ClearModp,N>::bit_decomposition_ClearModp(c)}
\func{Slice<ClearModp>::bit_decomposition_ClearModp(c, N)}
\func{Array<i64,N>::bit_decomposition_i64(v)}
\func{Slice<i64>::bit_decomposition_i64(v, N)}
Given a $\ClearModp$ value $c$ or an $\ints$ value $v$, these functions
compute the bit-decomposition of $c$ (resp. $v$) upto the $N$-th bit
position. Returning the result as an Array or Slice.

\func{Array<ClearModp,N>::bit_decomposition_ClearModp_Signed(c)}
\func{Slice<ClearModp>::bit_decomposition_ClearModp_Signed(c, N)}
The same but assumes the value $c$ is signed, i.e. if $p>2$ then
we bit-decompose the negative value $c-p$.

\subsection{Matrices}
\label{sec:matrices}
Again this data type is implemented using the SCALE memory.
It is thus only available for $\ints$, $\SecretI$, $\ClearModp$
and $\SecretModp$ data types.
To load the standard library version of Matrices use
\begin{lstlisting}
     use scale_std::matrix::Matrix;
\end{lstlisting}
There are a similar set of functions as for the Array type.

\func{Matrix::unitialized()}
To initialise a matrix with no pre-defined values, with
fives rows and four columns, use
\begin{lstlisting}
    let mut a= Matrix<SecretI64, 5, 4>::unitialized();
\end{lstlisting}


\func{a.fill(x)}
Fills an array, used with initialization
\begin{lstlisting}
    let a: Matrix<SecretI64, 5, 4> = Matrix::fill(zero);
\end{lstlisting}


\func{a.set(i,j,x)}
Does the obvious.

\func{a.get(i,j)}
Again does the obvious

\func{a.get_row(i)}
Does the obvious, returning the result as an \verb|Array|.

\func{a.get_column(i)}
Again does the obvious


\func{a.iter()}
Iterator over a \verb|Matrix|,
\begin{lstlisting}
    for (i, val) in a.iter().enumerate() {
        ....
    }
\end{lstlisting}
The iterator works in the order
$a_{0,0}, a_{0,1}, a_{0,2}, \ldots, a_{0,c-1}, a_{1,0}, a_{1,1}, \ldots, a_{r-1,c-1}$.

\noindent
To iterate backwards use
\begin{lstlisting}
    for (i, val) in a.iter().rev().enumerate() {
        ....
    }
\end{lstlisting}

\func{a.reveal()}
For the secret types this creates a non-secret version, as in.
\begin{lstlisting}
     let cint_matrix = sint_matrix.reveal();
\end{lstlisting}

\subsubsection{Stack Operations on Matrices}
One can also apply the stack operations on a matrix, which
applies to all the elements in the matrix.
The push operation works in the same order as the above
iterator, with the pop operations working in reverse.
The stack read operations (pop, peek etc) will pop and peek
the entire matrix, so you need to ensure the stack really has
that many elements on it; i.e. no access is performed outside
the stack, otherwise a runtime crash will occur.

\func{push(x), pop()}
\func{peek(i), poke(i, x)}
\func{peek_from_top(i), poke_from_top(i, x)}
These are called using the following convention
\begin{lstlisting}
    unsafe { Matrix::push(&a) }
    let a = unsafe { Matrix::<SecretI64, 5, 4>::pop() };
\end{lstlisting}

\subsection{ClearIEEE}
This class gives direct access to the IEEE-754 compliant implementation
of floating point numbers in SCALE.
The operations on \verb|ClearIEEE| values are usually much
faster than processing using \verb|ClearFixed| values, as
the basic arithmetic is implemented by passing the operation
(via a local function) to the SCALE engine; where it is implemented
in C++.
The usage is demonstrated in the example below:
\begin{lstlisting}
    use scale_std::ieee::*;
    #[inline(always)]
    #[scale::main(KAPPA = 40)]
    fn main() {
      let c= ClearIEEE::from(3.141592);
      print!(c,"\n");
    }
\end{lstlisting}

\func{ClearIEEE::from(x)}
You can convert an $\ints$, a \verb|ClearFixed<K,F>| 
or a constant \verb|f64| to a \verb|ClearIEEE| as follows:
\begin{lstlisting}
    let c= ClearIEEE::from(3.141592);

    let i: i64= 1677216;
    let fi=ClearIEEE::from(i);
\end{lstlisting}
You can also convert a  \verb|ClearIEEE| to an $\ints$, which rounds the
floating point value to a 64-bit integer (if possible).
\begin{lstlisting}
     let ic=i64::from(c);
\end{lstlisting}

\func{a.rep()}
The internal representation as an $\ints$ can be obtained by the
\verb|rep()| function.

\func{a.set(x)}
One can set the internal representation to a $\ints$ value using
the \verb|set()| function.

\func{ClearIEEE::NaN()}
Returns a representation of IEEE NaN.

\func{ClearIEEE::randomize()}
Produces a random value in the range $[0,\ldots,1)$.
This random number is the `same' for all players.
\begin{lstlisting}
   let a = ClearIEEE::randomize();
\end{lstlisting}

\subsubsection{Comparisons}
The \verb|ClearIEEE| type implements the \verb|Eq|, \verb|PartialEq|
and \verb|PartialOrd| traits. Therefore you can apply normal
comparison operators to \verb|ClearIEEE| types as in...
\begin{lstlisting}
    fn test_approx(a: ClearIEEE, val: ClearIEEE) -> i64
    {
        let lower = val - ClearIEEE::from(0.00001);
        let upper = val + ClearIEEE::from(0.00001);
        let ok = (a>lower) | (a<upper);
        ok as i64
    }
\end{lstlisting}
To aid with generic programming the following member functions
are also implemented, which output a value of type $\ints$,
\func{a.gt(x), a.ge(x), a.lt(x), a.le(x), a.gt(x), a.eq(x), a.ne(x)}
\func{a.gtz(), a.gez(), a.ltz(), a.lez(), a.gtz(), a.eqz(), a.nez(x)}


\subsection{SecretIEEE}
This class gives direct access to the IEEE-754 compliant implementation
of floating point numbers in SCALE. Recall these are implemented using
evaluations of binary circuits, so are not as fast as the \verb|sfloat|
types implemented in MAMBA (which are not currently available
in the Rust interface, but we hope to fix this soon).
The \verb|SecretFixed| types is more efficient than the \verb|SecretIEEE|
type and should usually be preferred. The benefit of the
\verb|SecretIEEE| type is that it is more expressive in terms
of the numbers ranges it can hold; but this comes at the drawback
of significant reduced performance.

\func{SecretIEEE::from(x)}
You can convert a $\SecretI$, a \verb|ClearIEEE| or a constant
\verb|f64|to a \verb|SecretIEEE|
as follows:
\begin{lstlisting}
    let c= ClearIEEE::from(3.141592);
    let sc=SecretIEEE::from(c);

    let i: i64= 1677216;
    let si= SecretI64::from(i);
    let sfi=SecretIEEE::from(si);
\end{lstlisting}
You can also convert a  \verb|SecretIEEE| to a $\SecretI$, which rounds the
floating point value to a 64-bit integer (if possible).
\begin{lstlisting}
     let ic=SecretI64::from(sc);
\end{lstlisting}


\func{a.rep()}
The internal representation as a $\SecretI$ can be obtained by the
\verb|rep()| function.

\func{a.set(x)}
One can set the internal representation to a $\SecretI$ value using
the \verb|set()| function.

\func{s.reveal()}
For the \verb|SecretIIEEE| type this creates a \verb|ClearIEEE| version, as in.
\begin{lstlisting}
     let ci = si.reveal();
\end{lstlisting}

\func{SecretIEEE::NaN()}
Returns a representation of IEEE NaN.

\func{SecretIEEE::randomize()}
Produces a (secret) random value in the range $[0,\ldots,1)$.
This random number is the `same' for all players.
\begin{lstlisting}
   let a = SecretIEEE::randomize();
\end{lstlisting}


\subsubsection{Comparisons}
The following `operators' can be applied between two \verb|SecretIEEE| values
The output is a $\SecretBit$ value.
As the result of the operator cannot be used in a conditional branch,
we use the member function notation for such `operators', as opposed
to the operator notation. Thus syntactically the programmer is less
likely to make a mistake.
\func{a.gt(x), a.ge(x), a.lt(x), a.le(x), a.gt(x), a.eq(x), a.ne(x)}

\noindent
The following give variants which compare just to zero.
\func{a.gtz(), a.gez(), a.ltz(), a.lez(), a.gtz(), a.eqz(), a.nez(x)}


\subsection{Arithmetic of ClearIEEE and SecretIEEE}
The following operations are allowed between two elements of type 
\verb|SecretIEEE| and \verb|ClearIEEE|.
\begin{verbatim}
     +   -   *   /
\end{verbatim}
Clear and secret values can be combined to result in a secret value.

\subsection{Bit Operations}
There are a number of routines to access and process Slices of bits,
which are represented in either $\ClearModp$ or $\SecretModp$ format.

\func{BitAdd(Slice<U> A, Slice<V> B)}
This produces a slice of $\SecretModp$ values corresponding to the
summation of the bits in $A$ and $B$. The two input slices are assumed
to be of the same length $k$, and the output slice is of length $k+1$.
The type of U and V can be either $\SecretModp$ or $\ClearModp$,
bit not both $\ClearModp$.
\begin{lstlisting}
    // 7 + 5 = 12
    let mut ca: Slice<ClearModp> = Slice::uninitialized(3);
    ca.set(0,&ClearModp::from(1));
    ca.set(1,&ClearModp::from(1));
    ca.set(2,&ClearModp::from(1));

    let mut sb: Slice<SecretModp> = Slice::uninitialized(3);
    sb.set(0,&SecretModp::from(1));
    sb.set(1,&SecretModp::from(0));
    sb.set(2,&SecretModp::from(1));

    let v_ca_sb = BitAdd(&ca,&sb);
\end{lstlisting}

\func{BitIncrement(Array<SecretModp, K> A)}
On input of an array of bits, represented as $\SecretModp$ values,
this function returns the slice of $\SecretModp$ values consting
of the incremeneted integer value.
\begin{lstlisting}
    let mut sa: Array<SecretModp, 3> = Array::uninitialized();
    sa.set(0,&SecretModp::from(1));
    sa.set(1,&SecretModp::from(1));
    sa.set(2,&SecretModp::from(1));
    let v_inc_sa = BitIncrement(&sa);
\end{lstlisting}

\func{BitLT(a, B, K)}
This takes a $\ClearModp$ value $B$ and an Slice of $\SecretModp$
values $A$, representing bits (of size at least $K$).
It computes the $\SecretModp$ value representing the conditional
$a < \sum_{i=0}^{K-1} b_i \cdot 2^i$.
\begin{lstlisting}
    let mut saa: Slice<SecretModp> = Slice::uninitialized(3);
    saa.set(0,&SecretModp::from(1));
    saa.set(1,&SecretModp::from(0));
    saa.set(2,&SecretModp::from(1));

    let four = ClearModp::from(4);
    let six = ClearModp::from(6);
    let cmp_four = BitLT(four, saa, 3);
    let cmp_six = BitLT(six, saa, 3);
\end{lstlisting}

\func{BitLTFull(Slice<U> A, Slice<V> B))}
The two input slices are assumed to be of the same length $K$, and 
the output is a single $\SecretModp$ value representing the conditional
$\sum_{i=0}^{K-1} a_i \cdot 2^i < \sum_{i=0}^{K-1} b_i \cdot 2^i$.
The type of U and V can be either $\SecretModp$ or $\ClearModp$,
bit not both $\ClearModp$.
This routine does not utilize any statistical security gap, thus
$K$ can be the size of $\log_2 p$.
\begin{lstlisting}
    let mut ca: Slice<ClearModp> = Slice::uninitialized(3);
    ca.set(0,&ClearModp::from(1));
    ca.set(1,&ClearModp::from(1));
    ca.set(2,&ClearModp::from(1));

    let mut sb: Slice<SecretModp> = Slice::uninitialized(3);
    sb.set(0,&SecretModp::from(1));
    sb.set(1,&SecretModp::from(0));
    sb.set(2,&SecretModp::from(1));

    let cmp_a1 = BitLTFull(&ca, &sb);
\end{lstlisting}

\func{BitDec::<K,M,KAPPA>(s)}
Given a (signed) $\SecretModp$ value which lies in $\Zk$
this produces the first $M$ bits of the bit-decomposition using 
statistical security parameter \verb|KAPPA|.
The output is given as a slice of $\SecretModp$ values, each
of which represents a bit.
\begin{lstlisting}
    let ss1 = SecretModp::from(5);
    let ss2 = SecretModp::from(-5);
    let v1 = BitDec::<10,5,40>(ss1);
    let v2 = BitDec::<10,5,40>(ss2);
\end{lstlisting}

\func{bitdec::<KAPPA>(s, k)}
Same but now $k$ can be a run time variable and not a compile time constant, 
and we use $K=M=k$.
\begin{lstlisting}
    let ss1 = SecretModp::from(5);
    let ss2 = SecretModp::from(-5);
    let v1 = bitdec::<40>(ss1, 10);
    let v2 = bitdec::<40>(ss2, 10);
\end{lstlisting}


\func{BitDecFull(x)}
Given a $\SecretModp$ value this produces the bit-decomposition
into a slice of dimension $\log_2 p$.
The value $x$ can be in the range $[0,\ldots,p)$ and no usage
is made of any statistical security parameter.
The output is given as a slice of $\SecretModp$ values, each
of which represents a bit.
\begin{lstlisting}
    let ss1 = SecretModp::from(5);
    let ss2 = SecretModp::from(-5);
    let v1 = BitDecFull(ss1);
    let v2 = BitDecFull(ss2);
\end{lstlisting}


\subsection{ClearInteger}
We define $\Zk$ as the set of integers $\{x \in \Z: -2^{k-1} \le x \le 2^{k-1}-1\}$,
which we embed into $\F_p$ via the map $x \mapsto x \pmod{p}$.
In MAMBA the user had to manually think of elements in $\F_p$
as representing values in $\Zk$, and manually keep track of their
sizes for all the `advanced' integer operations.
This was efficient, but prone to errors and misunderstandings.
In Rust we have a seperate type to represent elements in $\Zk$,
both in clear and secret form.
This is strongly typed in the sense that a
\verb|ClearInteger<3>|  is a different type from a
\verb|ClearInteger<4>|.

This class enables one to load and print clear integers, and acts
mainly as a helper class for the \verb|SecretInteger| class below.
The usage is demonstrated in the example below:
\begin{lstlisting}
    use scale_std::integer::*;
    #[inline(always)]
    #[scale::main(KAPPA = 40)]
    fn main() {
      let c: ClearInteger<3>= ClearInteger::from(2 as i64);
      print!(c,"\n");
    }
\end{lstlisting}

\func{ClearInteger::from(x)}
You can convert a $\ClearModp$ or a $\ints$ to a \verb|ClearInteger<K>| as follows:
\begin{lstlisting}
    let c = ClearModp::from(2);
    let ci: ClearInteger<3>= ClearInteger::from(c);
    let ci1: ClearInteger<3>= ClearInteger::from(2 as i64);
\end{lstlisting}

\func{a.recast()}
A value of one size can be recast to another using the \verb|recast|
member function. It is assumed the user knows what they are doing here
and that such recasting will not necessarily result in a valid representation
being created.
\begin{lstlisting}
    let c = ClearModp::from(2);
    let ci: ClearInteger<3>= ClearInteger::from(c);
    let ci1: ClearInteger<5>= unsafe{ a.recast() };
\end{lstlisting}

\func{a.rep()}
Returns the internal representation as a \verb|ClearModp| value.

\func{a.set(v)}
Sets the internal representation to be the \verb|ClearModp| value $v$.

\func{a.Trunc(M,S)}
Implements the deterministic trunction operation of rounding $a/2^M$.
When $S=\true$ the value $a$ is assumed to lie in $\Zk$,
but when $S=\false$ the value $a$ is assumed to lie in $[0,\ldots,2^{K-1})$.
\begin{lstlisting}
    let a: ClearInteger::<6> = ClearInteger::from(23);
    let b = a.Trunc(3, true);
\end{lstlisting}

\func{a.Mod2m(M,S)}
This computes the value of $a$ modulo $2^M$.
When $S=\true$ the value $a$ is assumed to lie in $\Zk$,
but when $S=\false$ the value $a$ is assumed to lie in $[0,\ldots,2^{K-1})$.
Mathematically this computes the expression
\[  \left( a + S \cdot 2^{K-1} \right) \pmod 2^M. \]
Implicitly this assumes that $M < K$.
\begin{lstlisting}
    let a: ClearInteger<10> = ClearInteger::from(-24);
    let b = a.Mod2m(5, true);
\end{lstlisting}

\func{a.Mod2(S)}
As above but takes $M=1$.


\func{ClearInteger::randomize()}
Produces a random value in $\Zk$.
This random number is the `same' for all players.
\begin{lstlisting}
   let a: ClearInteger<20> = ClearInteger::randomize();
\end{lstlisting}

\subsubsection{Comparisons}
The following `operators' can be applied between two \verb|ClearInteger<K>|
values,the output being a $\ClearModp$ value.
As the result of the operator cannot be used in a conditional branch,
we use the member function notation for such `operators', as opposed
to the operator notation. Thus syntactically the programmer is less
likely to make a mistake.
\func{a.gt(x), a.ge(x), a.lt(x), a.le(x), a.gt(x), a.eq(x), a.ne(x)}

\noindent
The following give variants which compare just to zero.
\func{a.gtz(), a.gez(), a.ltz(), a.lez(), a.gtz(), a.eqz(), a.nez(x)}




\subsection{SecretInteger}
Many operations on \verb|SecretInteger|s make use of the statistical
security gap \verb|KAPPA| defined in the pre-amble.

\func{SecretInteger::from(x)}
You can convert a $\ClearModp$ or a \verb|ClearInteger<K>| to a
\verb|SecretInteger<K>|
as follows:
\begin{lstlisting}
    let c = ClearModp::from(2);
    let ci: ClearInteger<3>= ClearInteger::from(c);
    let si1: SecretInteger<3>= SecretInteger::from(c);
    let si2: SecretInteger<3>= SecretInteger::from(ci);
\end{lstlisting}

\func{a.recast()}
A value of one size can be recast to another using the \verb|recast|
member function. It is assumed the user knows what they are doing here
and that such recasting will not necessarily result in a valid representation
being created.
\begin{lstlisting}
    let c = ClearModp::from(2);
    let si: SecretInteger<3>= SecretInteger::from(c);
    let si1: SecretInteger<5>= unsafe{ a.recast() };
\end{lstlisting}

\func{a.rep()}
Returns the internal representation as a \verb|SecretModp| value.

\func{a.set(v)}
Sets the internal representation to be the \verb|SecretModp| value $v$.

\func{a.reveal()}
For the \verb|SecretInteger| typs this creates a \verb|ClearInteger| version, as in.
\begin{lstlisting}
     let ci = si.reveal();
\end{lstlisting}

\func{a.Mod2m(M,S)}
This computes the value of $a$ modulo $2^M$.
When $S=\true$ the value $a$ is assumed to lie in $\Zk$,
but when $S=\false$ the value $a$ is assumed to lie in $[0,\ldots,2^{K-1})$.
Mathematically this computes the expression
\[  \left( a + S \cdot 2^{K-1} \right) \pmod 2^M. \]
Implicitly this assumes that $M < K$.
\begin{lstlisting}
    let a: SecretInteger<10> = SecretInteger::from(-24);
    let b = a.Mod2m(5, true);
\end{lstlisting}

\func{a.Mod2(S)}
As above but takes $M=1$.

\func{Pow2::<K,KAPPA>(b)}
If $b$ is a $\SecretModp$ in the range $[0,\ldots,K)$ this computes
the value $2^b$ as a \verb|SecretInteger<K>|
\begin{lstlisting}
    let b = SecretModp::from(28);
    let d = Pow2::<32,40>(b);
\end{lstlisting}

\func{a.TruncPr(M,S)}
Implements the probabilistic trunction operation of rounding $a/2^M$.
When $S=\true$ the value $a$ is assumed to lie in $\Zk$,
but when $S=\false$ the value $a$ is assumed to lie in $[0,\ldots,2^{K-1})$.
\begin{lstlisting}
    let a: SecretInteger::<6> = SecretInteger::from(23);
    let b = a.TruncPr(3, true);
\end{lstlisting}

\func{a.Trunc(M,S)}
Implements the deterministic trunction operation of rounding $a/2^M$.
When $S=\true$ the value $a$ is assumed to lie in $\Zk$,
but when $S=\false$ the value $a$ is assumed to lie in $[0,\ldots,2^{K-1})$.
\begin{lstlisting}
    let a: SecretInteger::<6> = SecretInteger::from(23);
    let b = a.Trunc(3, true);
\end{lstlisting}

\func{a.TruncRoundNearest(M,S)}
Implements the deterministic rounding to the nearest integer operation
of rounding $\lceil a/2^M \rfloor$.
When $S=\true$ the value $a$ is assumed to lie in $\Zk$,
but when $S=\false$ the value $a$ is assumed to lie in $[0,\ldots,2^{K-1})$.
\begin{lstlisting}
    let a: SecretInteger::<6> = SecretInteger::from(23);
    let b = a.TruncRoundNearest(3, true);
\end{lstlisting}

\func{a.ObliviousTrunc(m)}
Implements \verb|Trunc| operation where $m$ is a $\SecretModp$ value
which lies in the range $[0,\ldots,K)$. The output is an \verb|Array| of three
$\SecretModp$ values.
The first element is $t = \floor{a/2^m}$, the second is $a-2^m \cdot t$
and the third element is $2^m$.
\begin{lstlisting}
    let a = SecretModp::from(23);
    let b: SecretInteger::<6> = SecretInteger::from(a);
    let m = SecretModp::from(3);
    let c = b.ObliviousTrunc(m);
\end{lstlisting}


\func{SecretInteger::randomize()}
Produces a (secret) random value in $\Zk$.
This random number is the `same' for all players.
\begin{lstlisting}
   let a: SecretInteger<20> = SecretInteger::randomize();
\end{lstlisting}

\subsubsection{Comparisons}
The following `operators' can be applied between two \verb|SecretInteger<K>| 
values,the output being a $\SecretModp$ value.
As the result of the operator cannot be used in a conditional branch,
we use the member function notation for such `operators', as opposed
to the operator notation. Thus syntactically the programmer is less
likely to make a mistake.
\func{a.gt(x), a.ge(x), a.lt(x), a.le(x), a.gt(x), a.eq(x), a.ne(x)}

\noindent
The following give variants which compare just to zero.
\func{a.gtz(), a.gez(), a.ltz(), a.lez(), a.gtz(), a.eqz(), a.nez(x)}

\subsection{Arithmetic of ClearInteger and SecretInteger}
Since a reduction operation (to ensure the value really lies in
$\Zk$) is expensive we give function and operator versions of arithmetic.
\warning{In the member function versions no reduction occurs, thus 
these can be used when the user knows that no wrap around will occur.
We thus mark them as {\bf unsafe}}.
The operator versions allow for ease of use, and guarantee correctness,
but will be slower than the function versions.

The operators unarray minus (i.e. negation), addition, subtraction
and multiplication are supported. If a clear and secret value are
combined then the result is secret.
The equivalent function versions for arithmetic have the following signatures.
Note all are {\bf unsafe} bar the \verb|negate| function.
\begin{lstlisting}
  ClearInteger<K>::negate()                          -> ClearInteger<K>;
  ClearInteger<K>::add(ClearInteger<K>)              -> ClearInteger<K>;
  ClearInteger<K>::sub(ClearInteger<K>)              -> ClearInteger<K>;
  ClearInteger<K>::mul(ClearInteger<K>)              -> ClearInteger<K>;
  ClearInteger<K>::add_secret(SecretInteger<K>)      -> SecretInteger<K>;
  ClearInteger<K>::sub_secret(SecretClearInteger<K>) -> SecretInteger<K>;
  ClearInteger<K>::mul_secret(SecretInteger<K>)      -> SecretInteger<K>;
  SecretInteger<K>::negate()                         -> SecretInteger<K>;
  SecretInteger<K>::add(SecretInteger<K>)            -> SecretInteger<K>;
  SecretInteger<K>::sub(SecretInteger<K>)            -> SecretInteger<K>;
  SecretInteger<K>::mul(SecretInteger<K>)            -> SecretInteger<K>;
  SecretInteger<K>::add_clear(ClearInteger<K>)       -> SecretInteger<K>;
  SecretInteger<K>::sub_clear(ClearInteger<K>)       -> SecretInteger<K>;
  SecretInteger<K>::mul_clear(ClearInteger<K>)       -> SecretInteger<K>;
\end{lstlisting}




\subsection{ClearFixed}
A clear fixed point number is represented as a \verb|ClearInteger<K>| value $x$ and
a constant $F$, with the fixed point number so represented being $x/2^F$.
This representation, and the associated functions, should be avoided
if at all possible. The \verb|ClearIEEE| is much faster, providing
near native speed operations. 
The \verb|ClearFixed| type is really for interacting with the
\verb|SecretFixed| type. We provide a full math library for this type, 
but this again is mainly for interaction/testing purposes.

\begin{lstlisting}
    use scale_std::fixed_point::*;
    #[inline(always)]
    #[scale::main(KAPPA = 40)]
    fn main() {
      let c: ClearFixed<40,20>= ClearFixed::from(3.141592);
      print!(c,"\n");
    }
\end{lstlisting}

\func{ClearFixed::from(x)}
You can convert an \verb|f64| constant, an $\ints$,  a $\ClearModp$,
a \verb|ClearIEEE| or a \verb|ClearInteger<K>| to a \verb|ClearFixed<K,F>| as follows:
\begin{lstlisting}
    let two = ClearModp::from(2);
    let two_as_cfix: ClearFixed<40,20>= ClearFixed::from(c);
\end{lstlisting}

\func{a.recast()}
A value of one size can be recast to another using the \verb|recast|
member function. It is assumed the user knows what they are doing here
and that such recasting will necessarily result in a valid representation
being created.
\begin{lstlisting}
    let c = ClearModp::from(2);
    let ci: ClearFixed<40,20>= ClearFixed::from(c);
    let ci1: ClearFixed<50,30>= unsafe{ a.recast() };
\end{lstlisting}

\func{a.rep()}
Returns the internal representation as a \verb|ClearInteger<K>| value,
this is basically the fixed point value multiplied by $2^F$.

\func{a.rep_integer()}
This returns the \verb|ClearModp| value which is the integer part
of the fixed point number.
Basically it executes $x >> F$ if $x$ is positive and
$-((-x)>>F)$ if $x$ is negative.

\func{a.set(v)}
Sets the internal representation to be the \verb|ClearInteger<K>| value $v$.

\func{a.set_modp(v)}
Sets the internal representation to be the \verb|ClearInteger<K>| value obtianed by calling \verb|set| for the type
\verb|ClearInteger<K>| on the \verb|ClearModp| value $v$.


\func{ClearFixed::randomize()}
Produces a random value in the range $[0,\ldots,1)$, it
assumes that $K > F$.
This random number is the `same' for all players.
\begin{lstlisting}
   let a: ClearFixed<20,10> = ClearFixed::randomize();
\end{lstlisting}



\subsubsection{Comparisons}
The following `operators' can be applied between two \verb|ClearFixed<K,F>|
values,the output being a $\ClearModp$ value.
As the result of the operator cannot be used in a conditional branch,
we use the member function notation for such `operators', as opposed
to the operator notation. Thus syntactically the programmer is less
likely to make a mistake.
\func{a.gt(x), a.ge(x), a.lt(x), a.le(x), a.gt(x), a.eq(x), a.ne(x)}

\noindent
The following give variants which compare just to zero.
\func{a.gtz(), a.gez(), a.ltz(), a.lez(), a.gtz(), a.eqz(), a.nez(x)}



\subsection{SecretFixed}
A secret fixed point number is represented as a \verb|SecretInteger<K>| value $x$ and
a constant $F$, with the fixed point number so represented being $x/2^F$.
Many operations on \verb|SecretFixed| values make use of the statistical
security gap \verb|KAPPA| defined in the pre-amble.

\begin{lstlisting}
    use scale_std::fixed_point::*;
    #[inline(always)]
    #[scale::main(KAPPA = 40)]
    fn main() {
      let c: SecretFixed<40,20>= SecretFixed::from(3.141592);
      print!(c,"\n");
    }
\end{lstlisting}

\func{SecretFixed::from(x)}
You can convert an \verb|f64| constant, an $\ints$,  $\ClearModp$,
\verb|ClearInteger<K>| or a \verb|ClearFixed<K,F>| to a \verb|SecretFixed<K,F>| as follows:
\begin{lstlisting}
    let two = ClearModp::from(2);
    let two_as_cfix: ClearFixed<40,20>= ClearFixed::from(c);
    let two_as_sfix: SecretFixed<40,20>= SecretFixed::from(c);
\end{lstlisting}

\func{a.recast()}
A value of one size can be recast to another using the \verb|recast|
member function. It is assumed the user knows what they are doing here
and that such recasting will necessarily result in a valid representation
being created.
\begin{lstlisting}
    let c1: SecretFixed<40,20>= ClearFixed::from(c);
    let c2: SecretFixed<50,30>= unsafe{ a.recast() };
\end{lstlisting}

\func{a.rep()}
Returns the internal representation as a \verb|SecretInteger<K>| value,
this is basically the fixed point value multiplied by $2^F$.

\func{a.set(v)}
Sets the internal representation to be the \verb|SecretInteger<K>| value $v$.

\func{a.set_modp(v)}
Sets the internal representation to be the \verb|SecretInteger<K>| value obtianed by calling \verb|set| for the type
\verb|SecretInteger<K>| on the \verb|SecretModp| value $v$.


\func{SecretFixed::randomize()}
Produces a (secret) random value in the range $[0,\ldots,1)$, it
assumes that $K > F$.
This random number is the `same' for all players.
\begin{lstlisting}
   let a: SecretFixed<20,10> = SecretFixed::randomize();
\end{lstlisting}

\subsubsection{Comparisons}
The following `operators' can be applied between two \verb|SecretFixed<K,F>|
values,the output being a $\SecretModp$ value.
As the result of the operator cannot be used in a conditional branch,
we use the member function notation for such `operators', as opposed
to the operator notation. Thus syntactically the programmer is less
likely to make a mistake.
\func{a.gt(x), a.ge(x), a.lt(x), a.le(x), a.gt(x), a.eq(x), a.ne(x)}

\noindent
The following give variants which compare just to zero.
\func{a.gtz(), a.gez(), a.ltz(), a.lez(), a.gtz(), a.eqz(), a.nez(x)}


\subsection{Arithmetic of ClearFixed and SecretFixed}
Standard arithmetic operations $+, -, \times, /$ can be applied to
types \verb|ClearFixed<K,F>| and \verb|SecretFixed<K,F>|.
To ensure efficiency {\em no check is performed for overflow}.
Thus if the number exceed $2^{K-F-1}$ in absolute value then 
undefined behvaiour is likely to result.
Clear and secret values can be combined to result in a secret
value.

\subsection{ClearFloat}
A clear floating point number is represented by a an Array of five 
\verb|ClearModp|-elements: (v, p, z, s, err). 
They represent the value
\[ u=(1-2\cdot s) \cdot (1-z) \cdot v \cdot 2^{p}, \]
with the convention that if $z=1$ then $s=0$.
Note, that as Arrays do not implement \verb|Copy| neither does  \verb|ClearFloat|,
it does however implement \verb|Clone|.
The \verb|ClearFloat| type is meant for interacting with the \verb|SecretFloat| type. 
It is very slow in comparison to the \verb|ClearIEEE| and \verb|ClearFixed| types.

The type comes with two parameters $V$ and $P$, which describe the bitsize of the 
mantissa and the exponent respectively. 
An optional parameter is \verb|DETECT_OVERFLOW| (default value: $\true$), that indicates whether 
mathematical operators should check if their output exceeds the bounds of $V$ and $P$.
To set the  \verb|DETECT_OVERFLOW| value to $\false$ you should specify this in 
the macro before the main call, as in....
\begin{lstlisting}
    use scale_std::floating_point::*;
    #[inline(always)]
    #[scale::main(KAPPA = 40, DETECT_OVERFLOW = false)]
    fn main() {
      let c: ClearFloat<40,10>= ClearFloat::from(3.141592);
      print!(c,"\n");
    }
\end{lstlisting}
If the value is not defined in the initial macro then the default value of
$\true$ is taken:
\begin{lstlisting}
    use scale_std::floating_point::*;
    #[inline(always)]
    #[scale::main(KAPPA = 40)]
    fn main() {
      let c: ClearFloat<40,10>= ClearFloat::from(3.141592);
      print!(c,"\n");
    }
\end{lstlisting}
To avoid unnecessary cloning, the \verb|Print| trait also
allows references to a \verb|ClearFloat| to be passed to it:
\begin{lstlisting}
    use scale_std::floating_point::*;
    #[inline(always)]
    #[scale::main(KAPPA = 40)]
    fn main() {
      let c: ClearFloat<40,10>= ClearFloat::from(3.141592);
      print!(&c,"\n");
    }
\end{lstlisting}


\func{ClearFloat::from(x)}
You can convert a value of type \verb|f64|, $\ints$, \verb|ClearFixed<K,F>|,
\verb|ClearInteger<K>|\footnote{Note here $K$ does not necessarily have to
equal $V$.} or \verb|ClearIEEE| to \verb|ClearFloat|. 
\begin{lstlisting}
    let two_as_cfloat: ClearFixed<40,10>= ClearFloat::from(2_f64);
\end{lstlisting}
Note that the \verb|K| for \verb|ClearInteger<K>| and the \verb|K| for \verb|ClearFixed<K,F>| 
must be larger than the \verb|V| of \verb|ClearFloat<V,P>|.

\func{ClearFloat::set(x)}
Given an \verb|Array| of five $\ClearModp$ values this sets them 
to be the representation of a \verb|ClearFloat| value.

\func{a.v(), a.p(), a.z(), a.s(), a.err()}
Gets the associated representation value as a $\ClearModp$.

\subsubsection{Comparisons}
The following `operators' can be applied between two \verb|ClearFloat<V,P>|
values,the output being a $\ClearModp$ value.
As the result of the operator cannot be used in a conditional branch,
we use the member function notation for such `operators', as opposed
to the operator notation. Thus syntactically the programmer is less
likely to make a mistake.
\func{a.gt(x), a.ge(x), a.lt(x), a.le(x), a.gt(x), a.eq(x), a.ne(x)}

\noindent
The following give variants which compare just to zero.
\func{a.gtz(), a.gez(), a.ltz(), a.lez(), a.gtz(), a.eqz(), a.nez(x)}

To be able to avoid costly cloning, the comparisons can be used on both 
references and clones. All options for \verb|gt()| and \verb|gtz()| for  
\verb|ClearModp| are presented in following example:
\begin{lstlisting}
    let ca: ClearFloat<40,20> = ClearFloat::from(123546);
    let cb: ClearFloat<40,20> = ClearFloat::from(789);

    let gtcc = ca.clone().gt(cb.clone());
    let gtrc = (&ca).gt(cb.clone());
    let gtcr = ca.clone().gt(&cb);
    let gtrr = (&ca).gt(&cb);

    let gtzc = ca.clone().gtz();
    let gtzr = (&ca).gtz();
\end{lstlisting}


\subsection{SecretFloat}
A secret floating point number is represented by an Array of five 
\verb|SecretModp|-values. 
Together they represent the value as follows:
\[ u=(1-2\cdot s) \cdot (1-z) \cdot v \cdot 2^{p}, \]
again with the convention that if $z=1$ then $s=0$.
In this array, the mantissa $v$ has a bitsize \verb|V| and the exponent 
$p$ is of size \verb|P|. 
Many operations on \verb|SecretFloat| values make use of the statistical 
security gap \verb|KAPPA| defined in the pre-amble. 
As for the \verb|ClearFloat| type, there is also a parameter \verb|DETECT_OVERFLOW|
defined in the same manner.
Note, that as Arrays do not implement \verb|Copy| neither does  \verb|SecretFloat|,
it does however implement \verb|Clone|.


\begin{lstlisting}
    use scale_std::floating_point::*;
    #[inline(always)]
    #[scale::main(KAPPA = 40)]
    fn main() {
      let c: SecretFloat<40,20>= SecretFloat::from(3.141592);
      print!(c.reveal(),"\n");
    }
\end{lstlisting}

\func{SecretFloat::from(x)}
You can convert a value of type \verb|f64|, $\ints$, \verb|SecretInteger<K>| 
or \verb|SecretFixed<K,F>|  to 
\verb|SecretFloat| as follows:
\begin{lstlisting}
    let two_as_sfloat: SecretFloat<40,10> = SecretFloat::from(2_f64);
\end{lstlisting}
Note that the \verb|K| for \verb|SecretInteger<K>| and the \verb|K| for \verb|SecretFixed<K,F>| 
must be larger than the \verb|V| of \verb|SecretFloat<V,P>|.

\func{SecretFloat::set(x)}
Given an \verb|Array| of five $\SecretModp$ values this sets them
to be the representation of a \verb|SecretFloat| value.

\func{a.v(), a.p(), a.z(), a.s(), a.err()}
Gets the associated representation value as a $\SecretModp$.

\func{a.reveal()}
You can reveal a value of type \verb|SecretFloat| to a \verb|ClearFloat|-type value as follows:
\begin{lstlisting}
    let cfa: ClearFloat<40,10> = sfa.reveal();
\end{lstlisting}
If the entered value is invalid (err == 0), \verb|reveal()| returns a \verb|ClearFloat| with all parameter values set to zero. This to avoid information leakage.

\subsubsection{Comparisons}
The following `operators' can be applied between two \verb|SecretFloat<V,P>|
values,the output being a $\SecretModp$ value.
As the result of the operator cannot be used in a conditional branch,
we use the member function notation for such `operators', as opposed
to the operator notation. Thus syntactically the programmer is less
likely to make a mistake.
\func{a.gt(x), a.ge(x), a.lt(x), a.le(x), a.gt(x), a.eq(x), a.ne(x)}

\noindent
The following give variants which compare just to zero.
\func{a.gtz(), a.gez(), a.ltz(), a.lez(), a.gtz(), a.eqz(), a.nez(x)}

To be able to avoid costly cloning, the comparisons can be used on both 
references and clones. All options for \verb|gt()| and \verb|gtz()| for  
\verb|SecretModp| are presented in following example:
\begin{lstlisting}
    let sa: SecretFloat<40,20> = SecretFloat::from(123546);
    let sb: SecretFloat<40,20> = SecretFloat::from(789);

    let gtcc = sa.clone().gt(sb.clone());
    let gtrc = (&sa).gt(sb.clone());
    let gtcr = sa.clone().gt(&sb);
    let gtrr = (&sa).gt(&sb);

    let gtzc = sa.clone().gtz();
    let gtzr = (&sa).gtz();
\end{lstlisting}

\subsection{Arithmetic of ClearFixed and SecretFixed}
Standard arithmetic operations $+, -, \times, /$ can be applied to
types \verb|ClearFloat<V,P>| and \verb|SecretFloat<V,P>|.
Depending on the flag parameter \verb|DETECT_OVERFLOW|, the results 
of these operations is checked for overflow/underflow.
Clear and secret values can be combined to result in a secret
value.

To avoid the costly \verb|.clone()| operation on every input to arithmetic 
operations (and comparisons), the operators also accept references. This 
allows the variable to be used in arithmetic without ending its lifetime.
These are accessed as in the following example for addition.
\begin{lstlisting}
    let ca: ClearFloat<40,20>  = ClearFloat::from(123456);
    let sa: SecretFloat<40,20> = SecretFloat::from(ca.clone());
    let cb: ClearFloat<40,20>  = ClearFloat::from(789);
    let sb: SecretFloat<40,20> = SecretFloat::from(cb.clone());

    let sscr = sa.clone() + &sb;
    let ssrr = &sa + &sb;
    let ssrc = &sa + sb.clone();
    let sscc = sa.clone() + sb.clone();

    let sccr = sa.clone() + &cb;
    let scrr = &sa + &cb;
    let scrc = &sa + cb.clone();
    let sccc = sa.clone() + cb.clone();

    let cscr = ca.clone() + &sb;
    let csrr = &ca + &sb;
    let csrc = &ca + sb.clone();
    let cscc = ca.clone() + sb.clone();

    let cccr = ca.clone() + &cb;
    let ccrr = &ca + &cb;
    let ccrc = &ca + cb.clone();
    let cccc = ca.clone() + cb.clone();
\end{lstlisting}

\subsection{ORAM Operations}
Given a \verb|SecretModp| index value $i$ and an \verb|Array| or
\verb|Slice| $A$ one wants to either read the value $A[i]$, or
one wants to update $A[i]$ with a new value $v$. One would want
to do this without revealing the value $i$, or even any access
pattern.
This is enabled via a Demux data structure which provided a linear
time read and write access to the data structure in an oblivious way.
The key thing needed is an upper bound on the bit length of the
index value $i$, which is kind of known in any case, as $i$ must
be less than the length of the  \verb|Array| or \verb|Slice|.

Note these operations require a bit length of the base prime of
larger than 100.

\func{a.read_oram::<K,KAPPA>(i)}
This takes a secret index of bit length less than $K$ are does a
read into the array $a$ at position $i$.
The value $K$ must be such that $2^K$ is greater than or equal
to the length of $a$.
As an example, we present an example for \verb|Array|s,
an identical syntax works for \verb|Slice|s,
\begin{lstlisting}
    let mut arr : Array<SecretModp, 20> = Array::uninitialized();
    for i in 0..20 {
       arr.set(i, &SecretModp::from(i as i64));
    }

    let i = SecretModp::from(7);
    let v = arr.read_oram::<5,40>(i);
\end{lstlisting}

\func{a.write_oram::<K,KAPPA>(i, v)}
This takes a secret index of bit length less than $K$ are does a
write into the array $a$ of the value $v$ at position $i$.
The value $K$ must be such that $2^K$ is greater than or equal
to the length of $a$.
As an example, we present an example for \verb|Slice|s,
an identical syntax works for \verb|Array|s,
\begin{lstlisting}
    let mut sli : Slice<SecretModp> = Slice::uninitialized(20);
    for i in 0..20 {
       sli.set(i, &SecretModp::from(i as i64));
    }

    let i = SecretModp::from(13);
    let v = SecretModp::from(666);
    sli.write_oram::<5,40>(i, v);
\end{lstlisting}

\func{Demux::<K,KAPPA>(i)}
In order to allow efficient indexing at the same index, we also
allow the creation of the demux data externally.
This function produces a \verb|Slice| of \verb|SecretModp| values
of length at least $2^K$, which corresponds to the demux array of
the index $i$.
Note, the function outputs an \verb|Slice| which is bigger than 
needed. This is done in order to enable a memory efficient algorithm.

\func{demux::<KAPPA>(i, k)}
Same, but now $k$ is a runtime variable and not a compile time constant.

\func{a.read_with_demux(\&demux)}
This is the same as the \verb|read_oram| operation, but uses
the demux output.
Thus we can execute
\begin{lstlisting}
    let i = SecretModp::from(19);
    let demx = Demux::<5,40>(i);
    let va = arr.read_with_demux(&demx);
    let vs = sli.read_with_demux(&demx);
\end{lstlisting}


\func{a.write_with_demux(\&demux, v)}
This is the same as the \verb|write_oram| operation, but uses
the demux output.
Thus we can execute
\begin{lstlisting}
    let i = SecretModp::from(19);
    let demx = Demux::<5,40>(i);
    let v = SecretModp::from(666);
    let va = arr.write_with_demux(&demx, v);
    let vs = sli.write_with_demux(&demx, v);
\end{lstlisting}

\func{a.read_with_demux_and_offset(offset, \&demux)}
This is the same as the \verb|read_oram_with_demux| operation, but 
offsets the demux operation to start at position \verb|offset|.
No checking is performed as to whether this gives array out
of bounds errors.
\begin{lstlisting}
    let i = SecretModp::from(5);
    let demx = demux::<40>(i, 3);
    let va = arr.read_with_demux_and_offset(13, &demx);
    let vs = sli.read_with_demux_and_offset(13, &demx);
\end{lstlisting}


\func{a.write_with_demux_and_offset(offset, \&demux, v)}
This is the same as the \verb|write_oram_with_demux| operation, 
but offsets the demux operation to start at position \verb|offset|.
No checking is performed as to whether this gives array out
of bounds errors.
\begin{lstlisting}
    let i = SecretModp::from(5);
    let demx = demux::(i,3);
    let v = SecretModp::from(666);
    let va = arr.write_with_demux(13, &demx, v);
    let vs = sli.write_with_demux(13, &demx, v);
\end{lstlisting}



