\subsection{SecretBit}
This is the same as the $\sbit$ in the underlying SCALE virtual machine.

\subsubsection{Conversion of Data}
\func{SecretBit::from(a)}
This takes an integer value $a$ and loads it into a value
of type $\SecretBit$.
\begin{lstlisting}
    let sa = SecretBit::from(true);
\end{lstlisting}
You can also load a $\SecretModp$ value into the secret bit,
although it is up to the programmer to ensure that
the value really is a bit. If it is not a bit, then some
information can leak.
\begin{lstlisting}
    let cb = SecretModp::from(0);
    let sa = SecretBit::from(sa);
\end{lstlisting}

\subsubsection{Other Operations}

\func{SecretBit::randomize()}
Produces a (secret) random bit.
This random number is the `same' for all players.
\begin{lstlisting}
   let a = SecretBit::randomize();
\end{lstlisting}

\subsubsection{Memory Access}
There is no $\SecretBit$ memory recall!

\subsubsection{Testing Data}
\func{x.test()}
In the SCALE target this takes the $\SecretBit$ value $x$,
applies a reveal operation to it, and then writes the
resulting $\regint$ value into memory.

In the Test target this takes a value stored in the $\regint$
memory saved on the last SCALE invocation, and compares it to
$x$. If the two values are the same it prints the value, and the
line number in the rust file where this was executed.
Otherwise it aborts.

\func{x.test_value(y)}
Test whether the value held in $x$ is the same as the value held in $y$.
This is the same as \verb|x.test()| except the value is compared to
$y$ and not the value emulated in the test environment.
