\mainsection{Math Library}
Eventually we expect every floating point type to
have a full set of math functions. 
Remember the \verb|ClearIEEE| type is must faster than
the \verb|ClearFixed<K,F>| type, but the \verb|SecretFixed<K,F>| 
is generally much faster than the \verb|SecretIEEE| type.
To include the functions in the math library use
\begin{lstlisting}
    use scale_std::math::*;
\end{lstlisting}
This captures traits \verb|Floor|, \verb|FAbs|
and \verb|Sqrt| (implementing member functions
\verb|floor()|, \verb|fabs()| and \verb|sqrt()| respectively), 
plus also a supertrait called
\verb|Float| which captures much of the properties
common to all floating point types.

Currently the following are available (we comment the
function out if it is not yet fully tested/implemented):

\subsection{Generic Routines}

\func{poly_eval(poly, x)}
This generic function with signature
\begin{lstlisting}
  pub fn poly_eval<S, C, const N: u64>(poly: Array<C, N>, x: S) -> S
  where
    S: Float,
    S: Add<C, Output = S>,
    S: Mul<S, Output = S>,
    C: Float,
\end{lstlisting}
allows one to evaluate a polynomial in type \verb|C| 
(a clear floating point type) at a value of type 
\verb|S| (which is either the {\em same} clear floating
point type, or its {\em associated} secret variant).

\func{Pade(P, Q, x)}
This generic function with signature
\begin{lstlisting}
  pub fn Pade<S, C, const N: u64>
   (poly_p: Array<C, N>, poly_q: Array<C, N>, x: S) -> S
  where
    S: Float,
    S: Add<S, Output = S>,
    S: Mul<S, Output = S>,
    S: Div<S, Output = S>,
    S: Mul<C, Output = S>,
    S: From<C>,
    C: Float,
\end{lstlisting}
allows one to evaluate the Pade approximation given
by polynomials $P$ and $Q$ in type \verb|C|
(a clear floating point type) at a value of type
\verb|S| (which is either the {\em same} clear floating
point type, or its {\em associated} secret variant).
The polynomials $P$ and $Q$ must have the same degree.


All floating point types \verb|T| have predefined constants
\begin{lstlisting}
   T::two_pi();
   T::pi();
   T::half_pi();
   T::ln2();
   T::e();
\end{lstlisting}


\subsection{ClearIEEE}
The following are all implemented via a local function call
to the SCALE C++ runtime; thus are relatively fast.
\begin{lstlisting}
    let c = ClearIEEE::from(0.5431);
    let t = c.acos();
    let t = c.asin();
    let t = c.atan();
    let t = c.cos();
    let t = c.cosh();
    let t = c.sin();
    let t = c.sinh();
    let t = c.tan();
    let t = c.tanh();
    let t = c.exp();
    let t = c.exp2()
    let t = c.log();
    let t = c.log2();
    let t = c.log10();
    let t = c.ceil();
    let t = c.fabs();
    let t = c.floor();
    let t = c.sqrt();
\end{lstlisting}
Note, \verb|exp2()| computes the function $2^x$.

\subsection{SecretIEEE}
The following functions are implemented
\begin{lstlisting}
    let s = SecretIEEE::from(0.5431);
    let t = s.acos();
    let t = s.asin();
    let t = s.atan();
    let t = s.cos();
    let t = s.cosh();
    let t = s.sin();
    let t = s.sinh();
    let t = s.tan();
    let t = s.tanh();
    let t = s.exp();
    let t = s.exp2()
    let t = s.log();
    let t = s.log2();
    let t = s.log10();
    let t = s.ceil();
    let t = s.fabs();
    let t = s.floor();
    let t = s.sqrt();
\end{lstlisting}
Note logarithms of negative values result in undefined
behaviour.

\subsection{ClearFixed}
The following functions are implemented
\begin{lstlisting}
    let c: ClearFixed<40,20>= ClearFixed::from(0.5431);
    let t = c.acos();
    let t = c.asin();
    let t = c.atan();
    let t = c.cos();
    let t = c.cosh();
    let t = c.sin();
    let t = c.sinh();
    let t = c.tan();
    let t = c.tanh();
    let t = c.exp();
    let t = c.exp2()
    let t = c.log();
    let t = c.log2();
    let t = c.log10();
    let t = c.ceil();
    let t = c.fabs();
    let t = c.floor();
    let t = c.sqrt();
\end{lstlisting}

\subsection{SecretFixed}
The following functions are implemented
\begin{lstlisting}
    let s: SecretFixed<40,20>= SecretFixed::from(0.5431);
    let s = c.acos();
    let s = c.asin();
    let s = c.atan();
    let s = c.cos();
    let s = c.cosh();
    let s = c.sin();
    let s = c.sinh();
    let s = c.tan();
    let s = c.tanh();
    let s = c.exp();
    let s = c.exp2()
    let s = c.log();
    let s = c.log2();
    let s = c.log10();
    let s = c.ceil();
    let s = c.fabs();
    let s = c.floor();
    let s = c.sqrt();
\end{lstlisting}


\subsection{ClearFloat}
The following functions are implemented.
For the exponential and (hence) the hyperbolic functions we assume
that $V < 2^{P-1}$, which it will be in almost all instances;
unless $P$ is very small.
\begin{lstlisting}
    let c: ClearFloat<40,20>= ClearFloat::from(0.5431);
    // let t = c.clone().acos();
    // let t = c.clone().asin();
    // let t = c.clone().atan();
    // let t = c.clone().cos();
    let t = c.clone().cosh();
    // let t = c.clone().sin();
    let t = c.clone().sinh();
    // let t = c.clone().tan();
    let t = c.clone().tanh();
    let t = c.clone().exp();
    let t = c.clone().exp2()
    let t = c.clone().log();
    let t = c.clone().log2();
    let t = c.clone().log10();
    let t = c.clone().ceil();
    let t = c.clone().fabs();
    let t = c.clone().floor();
    let t = c.clone().sqrt();
\end{lstlisting}

\subsection{SecretFloat}
The following functions are implemented.
For the exponential and (hence) the hyperbolic functions we assume
that $V < 2^{P-1}$, which it will be in almost all instances;
unless $P$ is very small.
\begin{lstlisting}
    let s: SecretFloat<40,20>= SecretFloat::from(0.5431);
    // let t = s.clone().acos();
    // let t = s.clone().asin();
    // let t = s.clone().atan();
    // let t = s.clone().cos();
    let t = s.clone().cosh();
    // let t = s.clone().sin();
    let t = s.clone().sinh();
    // let t = s.clone().tan();
    let t = s.clone().tanh();
    let t = s.clone().exp();
    let t = s.clone().exp2()
    let t = s.clone().log();
    let t = s.clone().log2();
    let t = s.clone().log10();
    let t = s.clone().ceil();
    let t = s.clone().fabs();
    let t = s.clone().floor();
    let t = s.clone().sqrt();

\end{lstlisting}
